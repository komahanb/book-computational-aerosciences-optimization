\chapter{Experimental Equipment}
Lorem ipsum dolor sit amet, consectetur adipiscing elit, sed do eiusmod tempor incididunt ut labore et dolore magna aliqua. Ut enim ad minim veniam, quis nostrud exercitation ullamco laboris nisi ut aliquip ex ea commodo consequat. Duis aute irure dolor in reprehenderit in voluptate velit esse cillum dolore eu fugiat nulla pariatur. Excepteur sint occaecat cupidatat non proident, sunt in culpa qui officia deserunt mollit anim id est laborum.

\chapter{Data Processing}
Lorem ipsum dolor sit amet, consectetur adipiscing elit, sed do eiusmod tempor incididunt ut labore et dolore magna aliqua. Ut enim ad minim veniam, quis nostrud exercitation ullamco laboris nisi ut aliquip ex ea commodo consequat. Duis aute irure dolor in reprehenderit in voluptate velit esse cillum dolore eu fugiat nulla pariatur. Excepteur sint occaecat cupidatat non proident, sunt in culpa qui officia deserunt mollit anim id est laborum.

\chapter{\uppercase{Finite Element Procedure for Three-bar Truss Analysis}}
\label{appendix:3bar}

The finite element procedure adopted for obtaining the nodal displacements ($u_x$ and $u_y$) as well as the elemental stresses ($\sigma_1$, $\sigma_2$, and $\sigma_3$) for the three-bar truss problem (see section~\ref{truss}) is discussed here.
The structure is assumed to have three elements with two degrees of freedom at each node (\ie~ a two-dimensional truss analysis). The element connectivity information is given in Table~\ref{tab:connectivity}.
\begin{table}[H]
\centering
\caption{Connectivity of elements.}
\label{tab:connectivity}
\begin{tabular}{cccccc}
\hline\hline
Element & \mc{2}{Node} &Element length  & $\lambda_e=\cos(\phi_e)$ & $\mu_e=\sin(\phi_e)$\\
\cline{2-3} % horizontal lines connecting cols. 2-3, 5-6
{}& ${i}$ & $j$ &$l_e$  & & \\
\hline
1 &1 & 4 & $L_1={H}/{\sin(\phi_1)}$  & $\cos(\phi_1)$   &  $\sin(\phi_1)$  \\
2 &2 & 4 & $L_2={H}/{\sin(\phi_2)}$  & $\cos(\phi_2)$   & $\sin(\phi_2)$ \\
3 &3 & 4 & $L_3={H}/{\sin(\phi_3)}$  & $\cos(\phi_3)$   &  $\sin(\phi_3)$\\
\hline
\end{tabular}
\end{table}
%\begin{landscape}
\noindent The stiffness matrix of the $e$-th element in global coordinate system is given by,
\bea\label{eq:stiff}
\begin{split}
\kmat_e & =\frac{E_eA_e}{L_e}\left[\begin{array}[h!]{rrrr}
    {\lambda_e}^2 & \lambda_e\mu_e  & {-\lambda_e}^2 &  -\lambda_e\mu_e \\
    \lambda_e\mu_e &  {\mu_e}^2 &-\lambda_e\mu_e &- {\mu_e}^2\\
   - {\lambda_e}^2& -\lambda_e\mu_e &  {\lambda_e}^2 & \lambda_e\mu_e \\
    -\lambda_e\mu_e & - {\mu_e}^2 &\lambda_e\mu_e & {\mu_e}^2\end{array}\right].
\end{split}
\eea
\\
The individual stiffness matrices of each element: $\kmat_1$, $\kmat_2$, and $\kmat_3$ are obtained using Eq.~(\ref{eq:stiff}) and Table~\ref{tab:connectivity}. The elemental stiffness matrices are assembled to form the global stiffness matrix $\bm{K}$ of size $8 \times 8$ (not shown here).
After applying the boundary conditions (i.e. $x$- and $y$-displacements at nodes $1$, $2$ and $3$ are zero) and performing elimination,  a simplified linear system  is obtained:
$\bm{K}\bm{Q}=\bm{F}$, where $\bm{Q}=\left\{\begin{array}[h!]{rr}Q_{4x}\\ Q_{4y} \end{array}\right\}$, $\bm{F}=\left\{\begin{array}[h!]{rr}F_{4x}\\ F_{4y} \end{array}\right\}$ and $\bm{K}=\left[\begin{array}[h!]{rr}
\frac{E_1A_1}{L_1}{{\lambda_1}^2} +  \frac{E_2A_2}{L_2}{{\lambda_2}^2} + \frac{E_3A_3}{L_3}{{\lambda_3}^2}  &  \frac{E_1A_1}{L_1}{{\lambda_1}{\mu_1}} +  \frac{E_2A_2}{L_2}{{\lambda_2}{\mu_2}} + \frac{E_3A_3}{L_3}{{\lambda_3}{\mu_3}}\\
\frac{E_1A_1}{L_1}{{\lambda_1}{\mu_1}} +  \frac{E_2A_2}{L_2}{{\lambda_2}{\mu_2}} + \frac{E_3A_3}{L_3}{{\lambda_3}{\mu_3}} &   \frac{E_1A_1}{L_1}{{\mu_1}^2} +  \frac{E_2A_2}{L_2}{{\mu_2}^2} + \frac{E_3A_3}{L_3}{{\mu_3}^2}
\end{array}\right]$.\\\\
The x- and y-displacements at node $4$ are given by, $\bm{Q}=\bm{K}^{-1}\bm{F}$.
Once the nodal displacements are found, the stresses in element $i-j$ can be calculated using:
\bea\label{stresseq}
\sigma_e = \frac{E_e}{L_e}\left\{\begin{array}[h!]{cccc} -\lambda_e & -\mu_e & \lambda_e & \mu_e \end{array}\right\}~\left\{\begin{array}[h!]{rrrr} Q_{ix} \\ Q_{iy} \\ Q_{jx} \\ Q_{jy} \end{array}\right\}.
\eea
Using Eq.~\eqref{stresseq} and Table~\ref{tab:connectivity}, expressions for the stresses acting on each element can be obtained:
\bea\label{eq:stresses}
\begin{split}
  \sigma_1 &= \frac{E_1}{L_1}\left\{\begin{array}[h!]{cccc} -\lambda_1 & -\mu_1 & \lambda_1 & \mu_1 \end{array}\right\}~\left\{\begin{array}[h!]{rrrr} 0 \\ 0 \\ Q_{4x} \\ Q_{4y} \end{array}\right\}=\frac{E_1}{L_1}\left(Q_{4x}\lambda_1 + Q_{4y}\mu_1\right),\\
  \sigma_2 &= \frac{E_2}{L_2}\left\{\begin{array}[h!]{cccc} -\lambda_2 & -\mu_2 & \lambda_2 & \mu_2 \end{array}\right\}~\left\{\begin{array}[h!]{rrrr} 0 \\ 0 \\ Q_{4x} \\ Q_{4y} \end{array}\right\}=\frac{E_2}{L_2}\left(Q_{4x}\lambda_2 + Q_{4y}\mu_2\right),\\
  \sigma_3 &= \frac{E_3}{L_3}\left\{\begin{array}[h!]{cccc} -\lambda_3 & -\mu_3 & \lambda_3 & \mu_3 \end{array}\right\}~\left\{\begin{array}[h!]{rrrr} 0 \\ 0 \\ Q_{4x} \\ Q_{4y} \end{array}\right\}=\frac{E_3}{L_3}\left(Q_{4x}\lambda_3 + Q_{4y}\mu_3\right).
\end{split}
\eea
The constraints (for the optimization problem) can be evaluated by substituting Eq~(\ref{eq:stresses}) into Eq.~(\ref{eq:3bardet}). The gradients of the constraints and objective function with respect to the design variables are obtained via differentiation with Maple.
