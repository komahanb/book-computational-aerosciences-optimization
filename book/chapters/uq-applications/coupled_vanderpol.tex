\subsection{Van der Pol Oscillator}
%The Van der Pol oscillator, is a nonconservative stable oscillator, and was termed as a relaxation oscillator on its discovery. 
%This oscillator has been frequently employed for the investigation of the properties of nonlinear oscillators and various oscillatory phenomena in physical and biological systems.
The natural form governing differential equation for Van der Pol oscillator is
\begin{equation}\label{eqn:vanderpol-natural}
  {R}(t,{\ddot{u}(t)},{\dot{u}(t)},u(t)) := \ddot{u}(t) -\mu(1-{u(t)}^2) \cdot \dot{u}(t) + u(t) = 0 
\end{equation}
which is a scalar nonlinear second-order differential equation where $u(t), \dot{u}(t)$ and $\ddot{u}(t)$ are unknown scalar functions of independent parameter time $t$. 
%Since the system is autonomous the dependence of $t$ can be omitted without loss of generality. 
Let the initial conditions be $u(0)=1$ and $\dot{u}(0)=1$. 
The partial Jacobians required for linearization of~\eqref{eqn:vanderpol-natural} are
\begin{equation}
  \pd{R}{\ddot{u}}=1,\quad\pd{R}{\dot{u}} = -\mu \left(1-{u(t)}^2\right)\quad\mathrm{and}\quad\pd{R}{u}=1 + 2\mu u(t)\dot{u}(t).
\end{equation}
We consider the case where the oscillator parameter $\mu$ is a function of normally distributed random variable $y \sim {\mathcal{N}}{(1.0,0.25)}$.  
The stochastic nonlinear ODE is
\begin{equation}\label{eqn:vanderpol-natural-stochastic}
  {\mathcal{R}}(t,y,{\ddot{u}(t,y)},{\dot{u}(t,y)},u(t,y)) := \ddot{u}(t,y) -\mu(1-{u(t,y)}^2) \cdot \dot{u}(t,y) + u(t,y) = 0  
\end{equation}
The solution of the stochastic ODE using semi-intrusive projection is centered around
implicitly forming the stochastic residuals, Jacobians and initial conditions as described in Section~\ref{sec:semi-intrusive-sgm}.
The probabilistic moments determined using semi-intrusive projection are shown in Figures~\ref{fig:vpl-mean-variance} and ~\ref{fig:vpl-mean-std-band}.
\begin{figure}[h!]
  \centering
  \begin{subfigure}{0.49\textwidth}
    \includegraphics[width=\linewidth]{vpl-galerkin-expectation.pdf}
  \end{subfigure}
  \centering
  \begin{subfigure}{0.49\textwidth}
    \includegraphics[width=\linewidth]{vpl-galerkin-variance.pdf}
  \end{subfigure}
  \caption{Expectation (left) and variance (right) of solution of Van der Pol oscillator obtained using projection with $7$ terms.}
  \label{fig:vpl-mean-variance}
\end{figure}
\begin{figure}[h!]
  \centering
  \begin{subfigure}{\textwidth}
    \includegraphics[width=0.32\linewidth]{vpl-galerkin-one-sigma.pdf}
    \includegraphics[width=0.32\linewidth]{vpl-galerkin-two-sigma.pdf}
    \includegraphics[width=0.32\linewidth]{vpl-galerkin-three-sigma.pdf}
  \end{subfigure}
  \caption{The expected response quantities with overlaid bands of one (left), two (middle) and three (right) standard deviations using projection with $7$ terms.}
  \label{fig:vpl-mean-std-band}
\end{figure}

%% 
%% \clearpage
%% We consider two degree of freedom Van der Pol oscillator whose governing equations are written as 
%% \begin{equation}\label{eqn:vanderpol-coupled-residual}
%%   \begin{aligned}
%%     \ddot{u}_1 - \mu (1 - u_1^2)  \dot{u}_1 + u_1 & = 0 \\
%%     \ddot{u}_2 - \mu (1 - u_1^2)  \dot{u}_2 + u_2 & = 0 \\
%%   \end{aligned}
%% \end{equation}
%% In matrix form the residual of the equations are of the abstract form $R(t, \ddot{u}(t), \dot{u}(t),{u}(t))=0$, and can be written as
%% \begin{equation}
%%   \begin{aligned}
%%     \begin{bmatrix}
%%       1  & 0 \\
%%       0 & 1 \\
%%     \end{bmatrix}
%%     \begin{Bmatrix}
%%       \ddot{u}_1 \\
%%       \ddot{u}_2
%%     \end{Bmatrix} 
%%      - \mu 
%%     \begin{bmatrix}
%%       1 - u_1^2 & 0 \\
%%       0 & 1 - u_1^2 \\
%%     \end{bmatrix}
%%     \begin{Bmatrix}
%%       \dot{u}_1 \\
%%       \dot{u}_2
%%     \end{Bmatrix} 
%%      +
%%     \begin{bmatrix}
%%       1  & 0 \\
%%       0  & 1 \\
%%     \end{bmatrix}
%%     \begin{Bmatrix}
%%       {u}_1 \\
%%       {u}_2
%%     \end{Bmatrix} 
%%     =
%%     \begin{Bmatrix}
%%       0 \\
%%       0 \\
%%     \end{Bmatrix} 
%%   \end{aligned}
%% \end{equation}
%% To use the implicit time marching methods reported in Chapter~\ref{chapter:adjoint-ode}, we implement the deterministic Jacobians 
%% $\pd{R}{\ddot{u}} = 
%% \begin{bmatrix}
%%   1  & 0 \\
%%   0 & 1 \\
%% \end{bmatrix}$,
%% $\pd{R}{\dot{u}} = 
%% \begin{bmatrix}
%%   - \mu (1 - u_1^2)  & 0 \\
%%   0 & - \mu (1 - u_1^2) \\
%% \end{bmatrix}$,
%% and
%% $\pd{R}{u} = 
%% \begin{bmatrix}
%%   1 + 2 \mu u_1 \dot{u}_1 & 0 \\
%%   2 \mu u_1 \dot{u}_2   & 1 \\
%% \end{bmatrix}$.
%% Let the uncertain parameter $\mu(y) \sim {\cal{E}}(\mu=0,\beta=1)$,
%% and the states be $u_1 = u_1(t,y)$ and $u_2 = u_2(t,y)$.
%% 
%% We are interested in finding the mean and variance of the response due to the exponential distribution of the parameter $\mu$.
%% The stochastic sampling method to obtain these moments is to solve the deterministic equations~\eqref{eqn:vanderpol-coupled-residual} at sample values $\mu_q$ and computing the statistics.
%% The stochastic Galerkin method relies on a single time dependent solution of the stochastic governing equations.
%% Using the semi-intrusive stochastic Galerkin method, the stochastic residuals and Jacobians are formed by decomposing the deterministic residuals and Jacobians in terms of the probabilistic basis functions.
